% \iffalse meta-comment
% Copyright (C) 2009 by Martin Scharrer <martin@scharrer-online.de>
% http://www.scharrer-online.de/latex/
% -----------------------------------------------------------------
%
% This work may be distributed and/or modified under the
% conditions of the LaTeX Project Public License, either version 1.3
% of this license or (at your option) any later version.
% The latest version of this license is in
%   http://www.latex-project.org/lppl.txt
% and version 1.3 or later is part of all distributions of LaTeX
% version 2005/12/01 or later.
%
% This work has the LPPL maintenance status `maintained'.
%
% The Current Maintainer of this work is Martin Scharrer.
%
% This work consists of the file rcs-multi.dtx and the derived file 
% rcs-multi.sty.
% $Id$
% \fi
% \iffalse
%<*package|driver|wrapper>
\def\filedate$#1: #2 #3 #4-#5-#6 #7 ${%
 \year#4\month#5\day#6\relax
 \def\filedate{#4/#5/#6}%
 \def\filerev{#3}%
}
\filedate$Id$
\def\fileversion{v0.1}
%</package|driver|wrapper>
%<*driver>
\ProvidesFile{rcs-multi.dtx}
 [\filedate\space\fileversion\space RCS Keywords for multi-file LaTeX documents]

\documentclass{ltxdoc}
\usepackage{rcs-multi}
\usepackage{booktabs}
\usepackage{ifpdf}
\ifpdf
  % use hypdoc if you have it, hyperref else
  \IfFileExists{hypdoc.sty}
    {\usepackage{hypdoc}}
    {\usepackage{hyperref}}
\else
  \let\url\texttt
\fi
\usepackage{xspace}
\newcommand{\ie}{i.e.\@\xspace}
\newcommand{\eg}{e.g.\@\xspace}

\iffalse
\let\css=\cs
\let\op=\cs
\let\DescribeOption\DescribeMacro
\let\DescribeScript\DescribeOption
\else % crossreference of macros in documentation
\makeatletter

\usepackage{xspace}
\@namedef{seen@package@latex}{1} %^^A avoid footnotes for 'latex'
\newcommand*{\pkg}[1]{%
  \href{http://tug.ctan.org/pkg/#1}{\texttt{#1}}%
  % URL footnote (for print-out) on first appearance:
  \@ifundefined{seen@package@#1}{%
    \footnote{CTAN: \url{http://tug.ctan.org/pkg/#1}}%
    \@namedef{seen@package@#1}{1}%
  }{}%
  \xspace
}
\newcommand*{\rcsmulti}{%
  \texttt{rcs-multi}\xspace%
}

% link \cs to macro definitions
\let\origmacro\macro
\let\origendmacro\endmacro
\let\origStopEventually\StopEventually
\let\origPrintDescribeMacro\PrintDescribeMacro
\usepackage{xcolor}
\definecolor{darkred}{rgb}{0.333.0.0,0.0}
\hypersetup{colorlinks=true,linkcolor=darkred,urlcolor=darkred}
\definecolor{macrodesccolor}{rgb}{0.0,0.0,0.8}
\definecolor{macroimplcolor}{rgb}{0.0,0.0,0.4}
\definecolor{metacolor}{rgb}{0.0,0.4,0.4}
\definecolor{scriptcolor}{rgb}{0.2,0.6,0.2}
\definecolor{optioncolor}{rgb}{0.3.0.2,0}

\let\macroline\\
\newlength{\macrosep}
\setlength{\macrosep}{-3em}
\renewcommand{\meta@font@select}{\color{metacolor}\itshape}
\newcommand{\macroformat}[1]{\textbf{\ttfamily #1}}
\newcommand{\optionformat}[1]{\textbf{\sffamily #1}}
\newcommand{\scriptformat}[1]{\textbf{\ttfamily #1}}
\newcommand{\macroargformat}[1]{\texttt{#1}}
\newcommand{\scriptargformat}[1]{\textbf{#1}}
\newcommand{\macrohlinkprefix}{desc}
\newcommand{\macrolink}{}

\usepackage[T1]{fontenc}
\usepackage{lmodern}

\def\DescribeMacro{\@ifnextchar*{\DescribeMacroS}{\DescribeMacroN}}
\def\DescribeMacroN{%
  \bigskip\pagebreak[3]\par\noindent\DescribeMacroS*%
}
\def\DescribeMacroS*#1#2{%
  \begingroup
  \g@namedef{href@desc@#1}{}%
  \immediate\write\@mainaux{%
    \noexpand\g@namedef{href@desc@#1}{}%
  }%
  \@ifundefined{href@impl@#1}%
    {\let\macrolink\relax}%
    {\def\macrolink{\hyperlink{impl@#1}}}%
  \hypersetup{linkcolor=macrodesccolor}%
  \hspace*{\macrosep}%
  \raisebox{\baselineskip}[\baselineskip]{\hypertarget{desc@#1}{}}%
  \macrolink{\macroformat{\textcolor{macrodesccolor}{\textbackslash #1}}}%
  \noindent\mbox{}\macroargformat{#2}\nopagebreak
  \macroline*[0.2\baselineskip]%
  \endgroup
  \nopagebreak
  \ignorespaces
}
\def\DescribeScript{\@ifnextchar*{\DescribeScriptS}{\DescribeScriptN}}
\def\DescribeScriptN{%
  \bigskip\par\pagebreak[2]\noindent\DescribeScriptS*%
}
\def\DescribeScriptS*#1#2{%
  \hspace*{\macrosep}%
  \raisebox{\baselineskip}[\baselineskip]{\hypertarget{script@#1}{}}%
  \scriptformat{\textcolor{scriptcolor}{#1}}%
  \noindent\mbox{}\scriptargformat{\ {#2}}\macroline*[0.2\baselineskip]%
  \nopagebreak
}
\def\DescribeOption{\@ifnextchar*{\DescribeOptionS}{\DescribeOptionN}}
\def\DescribeOptionN{%
  \bigskip\par\noindent\DescribeOptionS*%
}
\def\DescribeOptionS*#1{%
  \hspace*{\macrosep}%
  \raisebox{\baselineskip}[\baselineskip]{\hypertarget{option@#1}{}}%
  \optionformat{\textcolor{optioncolor}{#1}}%
  \noindent\mbox{}\macroline*[0.2\baselineskip]%
  \nopagebreak
}
\newcounter{macrolevel}
\renewenvironment{macro}[1]{%
  \addtocounter{macrolevel}{1}%
  \expandafter\macroX\expandafter{\expandafter\@gobble\string#1}%
}{%
  \addtocounter{macrolevel}{-1}%
}
\providecommand*{\g@namedef}[1]{%
  \expandafter\gdef\csname #1\endcsname
}
\newcommand*{\macroX}[1]{%
  \ifnum\c@macrolevel<2
    \smallskip
  \fi
  \par\noindent
  \g@namedef{href@impl@#1}{}%
  \immediate\write\@mainaux{%
    \noexpand\g@namedef{href@impl@#1}{}%
  }%
  \@ifundefined{href@desc@#1}%
    {\let\macrolink\relax}%
    {\def\macrolink{\hyperlink{desc@#1}}}%
  \hspace*{\macrosep}%
  \raisebox{\baselineskip}[\baselineskip]{\hypertarget{impl@#1}{}}%
  \macrolink{\macroformat{%
    \textcolor{macroimplcolor}{\textbackslash #1}}}%
  \\*[\smallskipamount]%
  \@ifnextchar\begin{\vspace*{-\baselineskip}}{\imacroarg}%
}

\newcounter{macroargs}
\newcounter{nmacroarg}

\newcommand*{\imacroarg}[1][0]{%
  \setcounter{macroargs}{#1}%
  \setcounter{nmacroarg}{1}%
  \ifnum\c@macroargs>0
    \expandafter\imacroargX
  \fi
}
\newcommand*{\aftermacroargs}{%
  \@ifnextchar\begin
    {\\*[-2ex]\ignorespaces}%
    {\\*[\smallskipamount]\ignorespaces}%
}
\newcommand*{\imacroargX}[1]{%
  \hspace*{-1em}\texttt{\#\thenmacroarg:} #1\relax
  \ifnum\c@macroargs>1
    \newline
  \fi
  \addtocounter{nmacroarg}{1}%
  \addtocounter{macroargs}{-1}%
  \ifnum\c@macroargs>0
    \expandafter\imacroargX
  \else
    \expandafter\aftermacroargs
  \fi
}


\def\karg#1{\{\$\textcolor{metacolor}{#1}\$\}}
\def\kmarg#1{\{\$\meta{#1}\$\}}

\DeclareRobustCommand{\csi}[1]{%
  \begingroup
  \hypersetup{linkcolor=macroimplcolor}%
  \renewcommand{\macrohlinkprefix}{impl}%
  \@ifundefined{href@impl@#1}%
    {\let\macrolink\relax}%
    {\def\macrolink{\hyperlink{impl@#1}}}%
  \csX{#1}%
  \endgroup
}
\DeclareRobustCommand{\csd}[1]{%
  \begingroup
  \hypersetup{linkcolor=macrodesccolor}%
  \renewcommand{\macrohlinkprefix}{macro}%
  \@ifundefined{href@desc@#1}%
    {\let\macrolink\relax}%
    {\def\macrolink{\hyperlink{desc@#1}}}%
  \csX{#1}%
  \endgroup
}
\DeclareRobustCommand{\csX}[1]{%
  \begingroup
  \macrolink{\texttt{\textbackslash#1}}%
  \endgroup
}
\let\cs\csd
\DeclareRobustCommand{\css}[1]{\texttt{\textbackslash#1}}
\DeclareRobustCommand{\op}[1]{%
  \begingroup
  \hypersetup{linkcolor=optioncolor}%
  \hyperlink{option@#1}{\textbf{\sffamily #1}}%
  \endgroup
}
\DeclareRobustCommand{\scr}[1]{%
  \begingroup
  \hypersetup{linkcolor=scriptcolor}%
  \hyperlink{script@#1}{\scriptformat{#1}}%
  \endgroup
}

\def\StopEventually#1{\origStopEventually{#1}%
\let\cs\csi
}

\fi

\usepackage{graphicx}

\EnableCrossrefs
%\DisableCrossrefs
\CodelineIndex
%\PageIndex
\RecordChanges
%\OnlyDescription
\widowpenalty=500
\clubpenalty=500
\begin{document}
  \DocInput{rcs-multi.dtx}%
  \PrintChanges
  %\clearpage
  \PrintIndex
\end{document}
%</driver>
%<*package>
% \fi
%
% \CheckSum{0}
%
% {\makeatother
% \CharacterTable
%  {Upper-case    \A\B\C\D\E\F\G\H\I\J\K\L\M\N\O\P\Q\R\S\T\U\V\W\X\Y\Z
%   Lower-case    \a\b\c\d\e\f\g\h\i\j\k\l\m\n\o\p\q\r\s\t\u\v\w\x\y\z
%   Digits        \0\1\2\3\4\5\6\7\8\9
%   Exclamation   \!     Double quote  \"     Hash (number) \#
%   Dollar        \$     Percent       \%     Ampersand     \&
%   Acute accent  \'     Left paren    \(     Right paren   \)
%   Asterisk      \*     Plus          \+     Comma         \,
%   Minus         \-     Point         \.     Solidus       \/
%   Colon         \:     Semicolon     \;     Less than     \<
%   Equals        \=     Greater than  \>     Question mark \?
%   Commercial at \@     Left bracket  \[     Backslash     \\
%   Right bracket \]     Circumflex    \^     Underscore    \_
%   Grave accent  \`     Left brace    \{     Vertical bar  \|
%   Right brace   \}     Tilde         \~}
% }
% \changes{v0.1}{2009/03/23}{Initial version forked from rcs-multi v2.0}
%
% ^^A \GetFileInfo{rcs-multi.dtx}
%
% \DoNotIndex{\newcommand,\newenvironment,\AtBeginDocument,\AtEndDocument}
% \DoNotIndex{\def,\let,\edef,\xdef,\item,\space,\write,\jobname,\relax,\!}
% \DoNotIndex{\closeout,\csname,\DeclareRobustCommand,\else,\empty,\newwrite}
% \DoNotIndex{\endcsname,\expandafter,\fi,\Hurl,\hyper@normalise,\@ifnextchar}
% \DoNotIndex{\ifnum,\@ifundefined,\ifx,\immediate,\InputIfFileExists,\ }
% \DoNotIndex{\newcount,\noexpand,\openout,\PackageWarning,\@percentchar}
% \DoNotIndex{\@sanitize,\@makeother,\@iwrcs,\%,\_,\&,\^,\$,\#,\ ,\\,\if@filesw}
% \DoNotIndex{\gdef,\begingroup,\endgroup,\catcode}
% \DoNotIndex{\^,\ ,\_,\(,\),\$,\&,\#,\@ampersamchar,\AtEndOfPackage}
% \DoNotIndex{\@backslashchar,\begin,\bgroup,\chapter,\day}
% \DoNotIndex{\DeclareOption,\do,\dospecials,\@dottedtocline,\egroup}
% \DoNotIndex{\end,\ExecuteOptions,\filedate,\fileversion,\@for}
% \DoNotIndex{\futurelet,\g@addto@macro,\global,\@gobbletwo,\hline}
% \DoNotIndex{\hspace,\if@restonecol,\if@twocolumn,\ignorespaces}
% \DoNotIndex{\makeatletter,\MakeUppercase,\@mkboth,\month,\@namedef}
% \DoNotIndex{\NeedsTeXFormat,\newif,\onecolumn,\orig@fink@prepare}
% \DoNotIndex{\orig@fink@restore,\PackageError,\ProcessOptions}
% \DoNotIndex{\ProvidesPackage,\renewcommand,\RequirePackage}
% \DoNotIndex{\@restonecolfalse,\@restonecoltrue,\section,\strut}
% \DoNotIndex{\tableofcontents,\tableofrevisions,\texttt,\today}
% \DoNotIndex{\twocolumn,\@undefined,\url,\year}
% \DoNotIndex{\textwidth,\the,\string,\raggedright,\providecommand,\small,\toks@}
% \DoNotIndex{\medskipamount,\long,\leftskip,\clearpage,\advance,\addtolength}
% \DoNotIndex{\DeclareBoolOption,\DeclareStringOption,\DeclareVoidOption}
% \DoNotIndex{\ProcessKeyvalOptions,\SetupKeyvalOptions}
% \DoNotIndex{\@firstoftwo,\@secondoftwo,\@gobble}
%
% \title{The \textsf{rcs-multi} package}
% \author{Martin Scharrer \\ \url{martin@scharrer-online.de} \\
% \url{http://www.scharrer-online.de/latex/rcs-multi}\\
% CTAN: \url{http://tug.ctan.org/pkg/rcs-multi}}
% \date{Version \expandafter\@gobble\fileversion\\[0.5ex]\today}
%
% \ifpdf
% \hypersetup{%
%   pdfauthor   = {Martin Scharrer <martin@scharrer-online.de>},
%   pdftitle    = {The rcs-multi package, \fileversion, r\filerev\ from 
%   \filedate},
%   pdfsubject  = {Documentation of LaTeX package rcs-multi which allows the 
%   typesetting of RCS keywords in multi-file LaTeX documents},
%   pdfkeywords = {rcs-multi, LaTeX, RCS, multiple files, keywords, Version 
%   Control, Id}
% }%
% \fi
% \maketitle
%
% \section{Introduction}
% This package allows to typeset version control (VC) information provided by 
% RCS\footnote{RCS homepage: \url{http://www.gnu.org/software/rcs/}} keywords 
% (\eg |$||Id: ... $|) in \LaTeX\ documents which can contain of multiple |.tex| 
% files included using |\include| or |\input|.
%
% This package reads the keywords of all files and provides the VC information 
% of of the most recent changed file of the document to the user through a set 
% of macros. This information is written to the auxiliary |.aux| file during the 
% first \LaTeX\ run and read back at the next which introduces the same delay 
% known from the table of contents. The standard \LaTeX{} switch |\nofiles| can 
% be used to suppress the file generation.
%
% \subsection{Scope of Keywords}
% This package provides the RCS keyword data in two different scopes:
% document-global and file-local.
%
% \subsubsection*{Document Global}
% The document global macros, like \cs{rcsrev}, return the latest version 
% control information (keyword data) for the whole multi-file document, \ie the 
% information of the latest changed file of the document. To collect, sort and 
% provide this information is the main functionality of this package.
%
% \subsubsection*{Local to Current File}
% There are also file-local macros, \eg \cs{rcsfilerev}, which return the 
% version control information of the current file, \ie the file they are used 
% in. It is assumed here that every file using this macros calls first the macro
% \cs{rcsid}.  See section~\ref{sec:usage:id} for more details about this macro.  
% Please note that the file-local macros technically actually return the 
% \emph{last registered} information from the last \cs{rcsid}.
%
%
% \section{Usage}
% The version control information are provided by RCS keywords which
% first need to be read in by dedicated macros and can then be typeset using
% different macros.
%
% \subsection{Including RCS Keywords}\label{sec:usage:id}
% RCS keywords are included using \cs{rcsid}.  This macro should be written very 
% early in each file, \ie in the preamble of the main document soon after 
% |\documentclass| and |\usepackage{rcs-multi}| and as first in \emph{every} 
% subfile before an |\chapter| or similar macro. It does not create any output.  
% See section~\ref{sec:kwaccess} to learn how to typeset the keyword values.
%
% \DescribeMacro{rcsid}{\karg{Id}}
% \DescribeMacro*{rcsid}{\karg{Header}}
% The macro is for the |Id| keyword and must be written like shown.  A trailing 
% colon with or without spaces after the keyword name (`|Id|') is also valid but 
% \textbf{everything else} except a valid RCS string will cause a \TeX{} parse 
% error.
%
% \DescribeMacro{rcs}{\kmarg{keyword}}
% This macro let you typeset rcs keywords directly. The dollars will be stripped
% and the rest is typeset as normal text.
%
% \DescribeMacro{rcskwsave}{\kmarg{keyword}}
% This macro lets you include and save any keyword you like. The keyword can be
% already expanded or not (no value and only ``|:|'' or nothing after the key
% name). This macro is also used internally and does not create any output.
% Please note that the argument is read verbatim and that there should be no
% space between the macro and the argument's left brace.
%
%
% \subsection{Typesetting the Keyword Values}\label{sec:kwaccess}
% The following macros can be used to typeset the keyword values anywhere in the
% document. Please note that not all \LaTeX{} fonts have all special
% characters, \eg `|_|' is not provided in the standard roman font. To proper
% typeset file names and URLs containing these letters you can use either
% teletype font (|\texttt|) or use |{\urlstyle{rm}\rcsnolinkurl{...}}| which
% requires the \pkg{hyperref} package.
%
% \DescribeMacro{rcsrev}{}
% \DescribeMacro*{rcsdate}{}
% \DescribeMacro*{rcsauthor}{}
% These macros hold the keyword values of the whole document, \ie of the most
% recent revision. They can be used everywhere in every file of the \LaTeX{}
% document, after |\usepackage{rcs-multi}| of course. Please see
% section~\ref{sec:date} how to typeset parts of the date.
%
% \DescribeMacro{rcsfilerev}{}
% \DescribeMacro*{rcsfiledate}{}
% \DescribeMacro*{rcsfileauthor}{}
% These macros hold the keyword values of the current \LaTeX{} file, but only if
% it contains a \cs{rcsid} or \cs{rcsidlong} macro. Otherwise the macros hold
% either zero values or the values of the last file dependent on whether an
% option is enabled which enabled the \pkg{fink} package. Please see
% section~\ref{sec:date} how to typeset parts of the date. See \cs{rcskw} below
% for all other keywords.
%
% \DescribeMacro{rcsmainfilename}{}
% The macro \cs{rcsmainfilename} hold the filename of the main \LaTeX{file}. 
% It can be used to typeset this information anywhere in the document which 
% might be more descriptive as the name of the current file (which can be 
% typeset with \cs{rcskw}|{HeadURL}| or \cs{rcskw}|{Filename}| after \cs{rcsid} 
% or \cs{rcsidlong}, respectively). ^^A TODO: Change to correct names!
%
% \DescribeMacro{rcssetmainfile}{}
% This will declare the current file as the main LaTeX file by defining the 
% above macros. It will automatically be called at the end of the preamble so 
% the user normally doesn't have to use it by him- or herself as long it isn't 
% needed in the preamble.\par Please note that this macro changes the definition 
% of \cs{rcsmainfilename} directly without going over the auxiliary file.  
% Calling it in several files will make this two macros inconsistent.
%
% \DescribeMacro{rcskw}{\marg{keyword name}}
% All keywords saved with \cs{rcsid} or \cs{rcskwsave} can be typeset by this 
% macro which is a holdover from a very early version of this 
% package when multiple files where not supported.  It takes one argument which 
% must be a RCS keyword name. It then returns the current value of this 
% keyword or nothing (|\relax|) when the keyword was not set yet.
% Examples:\\
% \indent\indent |\textsl{Revision: \rcskw{Revision}}|\\
% \indent\indent |URL: \url{\rcskw{HeadURL}}|\\ ^^A TODO: Change name!
% In the second example |\url| (\pkg{hyperref} package) is used to add a hyperlink
% and to avoid problems with underscores (|_|) inside the URL.  \rcsmulti is
% also providing a macro \cs{rcsnolinkurl} which works like |\url| but doesn't
% adds an hyperlink. See the description of this macro for more details.
%
% If the given keyword doesn't exists a package warning is given to allow
% spelling errors to be tracked down. This doesn't work well when \cs{rcskw} is
% used inside |\url|. In this case the warning code will be typeset(!) verbatim
% into the document by |\url|.
%
% \DescribeMacro{rcskwdef}{\marg{keyword name}\marg{value}}
% This macro is used to define the keyword values. This is normally only called
% internally but could be used by the user to override single keywords.  The
% values can then be typeset by \cs{rcskw}.  Note that this macro has no
% influence on the calculation of the latest revision.
%
% \subsection{Accessing Date Values}\label{sec:date}
% \begin{tabular}{@{}l@{\hspace{-2\macrosep}}ll@{}}\\
% \DescribeMacro*{rcsyear}{}&
% \DescribeMacro*{rcsfileyear}{}\\
% \DescribeMacro*{rcsmonth}{}&
% \DescribeMacro*{rcsfilemonth}{}\\
% \DescribeMacro*{rcsday}{}&
% \DescribeMacro*{rcsfileday}{}\\
% \DescribeMacro*{rcshour}{}&
% \DescribeMacro*{rcsfilehour}{}\\
% \DescribeMacro*{rcsminute}{}&
% \DescribeMacro*{rcsfileminute}{}\\
% \DescribeMacro*{rcssecond}{}&
% \DescribeMacro*{rcsfilesecond}{}\\
% \end{tabular}
% \\*[\medskipamount]
% Whenever the date information is read, \ie by 
% \cs{rcskwsave}|{Date}| or \cs{rcsid}, the following macros are set to the 
% appropriate date parts for the 
% current file (the |\rcsfile...| versions) and for the whole document.
%
%
% \DescribeMacro{rcstime}{}
% \DescribeMacro*{rcsfiletime}{}
% This macros return the time part of the date only and simply return the
% corresponding hour, minute and second macros with a colon as separator.
%
% \DescribeMacro{rcspdfdate}{}
% Returns the last changed date of the whole document in a format needed for
% |\pdfinfo|. Can be used like this:\\
% \hbox{}\hfill|\pdfinfo{ /CreationDate (D:\rcspdfdate) }|\hfill\hbox{}\\
% to set the PDF creation date to the last changed date if you use |pdflatex| to
% compile your \LaTeX{} document.
%
% \DescribeMacro{rcstoday}{}
% \DescribeMacro*{rcsfiletoday}{}
% These macros typeset the document-global or current-file, respectively, using 
% the format of |\today| which depends on the used language.
% To adjust the language of your document use the \pkg{babel} package.
%
% \subsection{Using Full Author Names}
% If you like to have the full author\footnote{This means RCS authors,
% \eg the persons who commit changes into the rcs repository.} names, not only
% the usernames, in your document you can use the following macros. First you
% have to register all authors of the document with \cs{rcsRegisterAuthor} and
% then you can write \eg |\rcsFullAuthor{\rcsauthor}| or
% |\rcsFullAuthor{\rcsfileauthor}|.
%
% \DescribeMacro{rcsRegisterAuthor}{\marg{author}\marg{full name}}
% This macro registers \meta{full name} as full name for \meta{author} (a
% RCS username) for later use with \cs{rcsFullAuthor}.
%
% \DescribeMacro{rcsFullAuthor}{\marg{author name or macro}}
% \DescribeMacro*{rcsFullAuthor*}{\marg{author name or macro}}
% Takes the username as argument and returns the full name if it was registered
% first with \cs{rcsRegisterAuthor}, otherwise it returns the given username.
% The star version returns the username in parentheses after the full name.
% This is normally used in one of the following forms:\\
% \hspace*{3em}\cs{rcsFullAuthor}|{|\cs{rcsauthor}|}|\\
% \hspace*{3em}\cs{rcsFullAuthor}|{|\cs{rcsfileauthor}|}|\\
%
% \subsection{Using Full Revision Names}
% Like the author's also revision names/tags can be registered and used later.
% These macros were implemented on user request and have the drawback that you
% have to guess the next revision number of your document in order to get
% correct results when you like to tag the to-be-checked-in revision.  Please
% note that this has nothing to do with the normal tagging.
%
% \DescribeMacro{rcsRegisterRevision}{\marg{revision number}\marg{tag name}}
% This registers \meta{tag name} as tag name for \meta{revision number} for
% later use with \cs{rcsFullRevision}.
%
% \DescribeMacro{rcsFullRevision}{\marg{revision number or macro}}
% \DescribeMacro*{rcsFullRevision*}{\marg{revision number or macro}}
% Takes a revision number coming from a macro like \cs{rcsrev}, \cs{rcsfilerev}
% or a number as argument and returns the full name if it was registered first
% with \cs{rcsRegisterRevision}, otherwise it returns ``Revision \meta{revision
% number}''.  The star version returns also the revision number leaded by `r' in
% parentheses after the tag name, \eg |Name (1.2)|.
%
% \subsection{Verbatim URLs with and without Hyperlinks}
% \vspace{-\baselineskip}
% \DescribeMacro{rcsnolinkurl}{\marg{macro with returns special text}}
% This macro allows you to write |\rcsnolinkurl{\rcskw{HeadURL}}| and get the
% Head URL typeset verbatim. However |\url{|\cs{rcskw}|{HeadURL}}|
% (\pkg{hyperref} package) gives you the same result with a hyperlink. Both
% macros require the \pkg{hyperref} package which is not automatically loaded by
% \rcsmulti.  Please load it manually when you like to use \cs{rcsnolinkurl}.
%
% Please note that you can't use \pkg{hyperref}'s |\nolinkurl| because it won't
% expand \cs{rcskw}.
%
% \StopEventually{}
% %%%%%%%%%%%%%%%%%%%%%%%%%%%%%%%%%%%%%%%%%%%%%%%%%%%%%%%%%%%%%%%%%%%%%%%%%%%%
% \section{Implementation}
% \subsection{Package Header}
%    \begin{macrocode}
\NeedsTeXFormat{LaTeX2e}[1999/12/01]
\ProvidesPackage{rcs-multi}
 [\filedate\space\fileversion\space RCS Keywords for multi-file LaTeX documents]
%    \end{macrocode}

% \subsection{General Internal Macros}
% Some internal used macro which don't fit in any other section.
%
% \begin{macro}{\rcs@ifempty}[1]{string}
% Tests if the given argument is empty. If so the first of the next two token
% will be expanded, the second one otherwise.
%    \begin{macrocode}
\def\rcs@ifempty#1{%
  \begingroup
  \edef\rcs@temp{#1}%
  \ifx\rcs@temp\empty
    \endgroup
    \expandafter
    \@firstoftwo
  \else
    \endgroup
    \expandafter
    \@secondoftwo
  \fi
}
%    \end{macrocode}
% \end{macro}

% \begin{macro}{\rcs@ifequal}[2]{string a}{string b}
% Tests if the given arguments are identical, \eg same strings. If so the first
% of the next two token will be expanded, the second one otherwise.
%    \begin{macrocode}
\def\rcs@ifequal#1#2{%
  \begingroup
  \edef\rcs@stringa{#1}%
  \edef\rcs@stringb{#2}%
  \ifx\rcs@stringa\rcs@stringb
    \endgroup
    \expandafter
    \@firstoftwo
  \else
    \endgroup
    \expandafter
    \@secondoftwo
  \fi
}
%    \end{macrocode}
% \end{macro}

% \begin{macro}{\rcs@ifvalidrev}[1]{macro name}
% Checks if the given macro (by name) is a valid revision, \ie defined and
% greater than zero.
%    \begin{macrocode}
\def\rcs@ifvalidrev#1{%
  \begingroup
  \@ifundefined{#1}%
    {\def\rcs@temp{-1}}%
    {\expandafter\edef
     \expandafter\rcs@temp\expandafter{\csname #1\endcsname}}%
  \ifnum\rcs@temp>-1\relax
    \endgroup
    \expandafter
    \@firstoftwo
  \else
    \endgroup
    \expandafter
    \@secondoftwo
  \fi
}
%    \end{macrocode}
% \end{macro}

% \subsection{Definition of init values}
%    \begin{macrocode}
% Init values
\def\rcsrev{0.0}            \def\@rcs@rev{0.0}
\def\rcsdate{}              \def\@rcs@date{}
\def\rcsauthor{}            \def\@rcs@author{}
\def\rcsyear{0000}          \def\@rcs@year{0000}
\def\rcsmonth{00}           \def\@rcs@month{00}
\def\rcsday{00}             \def\@rcs@day{00}
\def\rcshour{00}            \def\@rcs@hour{00}
\def\rcsminute{00}          \def\@rcs@minute{00}
\def\rcssecond{00}          \def\@rcs@second{00}
\def\rcsmainfilename{}
\def\rcsmainurl{\rcsmainfilename}
\def\rcs@temp{}
\def\rcs@lastkw{}

\def\rcsfilerev{0.0}
\def\rcsfiledate{}
\def\rcsfileauthor{}
\def\rcsfileyear{0000}
\def\rcsfilemonth{00}
\def\rcsfileday{00}
\def\rcsfilehour{00}
\def\rcsfileminute{00}
\def\rcsfilesecond{00}
\def\rcsfileurl{}
\def\rcsfilename{}
%    \end{macrocode}
%

% \subsection{Time and \textit{Today} macros}
%
% \begin{macro}{\rcstime}
% \begin{macro}{\rcsfiletime}
% This macros simple use the hour, minute and second macros.
%    \begin{macrocode}
\def\rcstime{\rcshour:\rcsminute:\rcssecond}
\def\rcsfiletime{\rcsfilehour:\rcsfileminute:\rcsfilesecond}
%    \end{macrocode}
% \end{macro}
% \end{macro}

% These macros use the |\today| macro to typeset the current date using the
% local language settings. Thanks and credit goes to Manuel P\'egouri\'e-Gonnard
% for suggesting this feature and for providing some code.
% \begin{macro}{\rcstoday}
%    \begin{macrocode}
\newcommand*{\rcstoday}{%
  \begingroup
    \year\rcsyear \month\rcsmonth \day\rcsday
    \relax \today
  \endgroup
}
%    \end{macrocode}
% \end{macro}
%
% \begin{macro}{\rcsfiletoday}
%    \begin{macrocode}
\newcommand*{\rcsfiletoday}{%
  \begingroup
    \year\rcsfileyear \month\rcsfilemonth \day\rcsfileday
    \relax \today
  \endgroup
}
%    \end{macrocode}
% \end{macro}

% \subsection{Id macro}
% \begin{macro}{\rcsid}
% Calls \cs{rcskwsave} with |\@rcsidswtrue| so that the Id keyword will be
% parsed at the end of \cs{rcskwsave}.
%    \begin{macrocode}
\newcommand*{\rcsid}{%
  \@rcsidswtrue
  \rcskwsave
}
\newif\if@rcsidsw
\@rcsidswfalse
%    \end{macrocode}
% \end{macro}
%

% \begin{macro}{\rcs@scanId}[6]{file name}{revision}{date (YYYY-MM-DD)}{time
% (HH:MM:SS)}{author (username)}{rest}
% Scans rcs Id (after it got parsed by \cs{rcskwsave}).  Awaits only Id value
% without leading `|Id:|' and a trailing |\relax| as end marker.  It calls
% \cs{@rcs@scandate} to extract the date information and \cs{@rcs@updateid} to
% update global Id values and also sets the appropriate keywords.
%    \begin{macrocode}
\def\rcs@scanId#1,v #2 #3 #4 #5 #6\relax{%
  \@rcs@scandate{#3 #4}%
  \@rcs@updateid{#2}{#3 #4}{#5}{#1}%
  \rcskwdef{Filename}{#1}%
  \rcskwdef{Date}{#3 #4}%
  \rcskwdef{Revision}{#2}%
  \rcskwdef{Author}{#5}%
}
%    \end{macrocode}
% \end{macro}
%

% \begin{macro}{\@rcs@updateid}[4]{rev}{date}{author (username)}{url}
% We first define the expanded arguments to variables for the user.  The
% expansion is needed because the arguments content is mostly generic like
% |\rcs@value| which can change very soon after this macro.
%    \begin{macrocode}
\def\@rcs@updateid#1#2#3#4{%
  \xdef\rcsfilerev{#1}%
  \xdef\rcsfiledate{#2}%
  \xdef\rcsfileauthor{#3}%
  \xdef\rcsfileurl{#4}%
  \rcs@getfilename\rcsfileurl
%    \end{macrocode}
% Then we check if the revision is non-empty (not yet expanded by RCS?)
% and larger then the current maximum value |\@rcs@rev|.  If yes we save all
% value to save them in the .aux-file later.
%    \begin{macrocode}
  \ifx\rcsfiledate\empty\else
    \begingroup
      \edef\@tempa{\@rcs@year\@rcs@month\@rcs@day}
      \edef\@tempb{\rcsfileyear\rcsfilemonth\rcsfileday}
      \ifnum\@tempa<\@tempb
        \rcs@update
      \else\ifnum\@tempa=\@tempb
        \edef\@tempa{\@rcs@hour\@rcs@minute\@rcs@second}
        \edef\@tempb{\rcsfilehour\rcsfileminute\rcsfilesecond}
        \ifnum\@tempa<\@tempb
          \rcs@update
        \fi
      \fi\fi
    \endgroup
  \fi
}
%    \end{macrocode}
% \end{macro}
%

% \begin{macro}{\rcs@updateid}
% Updates the max-hold macros with the values of the current file VC 
% information.
%    \begin{macrocode}
\def\rcs@update{%
  \xdef\@rcs@rev{\rcsfilerev}%
  \xdef\@rcs@date{\rcsfiledate}%
  \xdef\@rcs@author{\rcsfileauthor}%
  \xdef\@rcs@year{\rcsfileyear}%
  \xdef\@rcs@month{\rcsfilemonth}%
  \xdef\@rcs@day{\rcsfileday}%
  \xdef\@rcs@hour{\rcsfilehour}%
  \xdef\@rcs@minute{\rcsfileminute}%
  \xdef\@rcs@second{\rcsfilesecond}%
  \xdef\@rcs@name{\rcsfilename}%
  \xdef\@rcs@url{\rcsfileurl}%
}
%    \end{macrocode}
% \end{macro}
%

% \begin{macro}{\rcs@catcodes}
% Changes all \TeX-special character to category ``other''. The newline aka
% return is changed to category ``ignore'' so line breaks are not taken as part
% of the verbatim arguments.
%    \begin{macrocode}
\def\rcs@catcodes{%
  \let\do\@makeother
  \dospecials
  \catcode`\^^M9
  \catcode`\ 10
  \catcode`\{1
  \catcode`\}2
}
%    \end{macrocode}
% \end{macro}

% \begin{macro}{\rcs@gdefverb}[1]{macro}
%    \begin{macrocode}
\def\rcs@gdefverb#1{%
  \begingroup
    \def\rcs@temp{#1}%
    \begingroup
      \rcs@catcodes
      \rcs@gdefverb@
}
%    \end{macrocode}
% \end{macro}

% \begin{macro}{\rcs@defverb@}[1]{verbatim stuff}
%    \begin{macrocode}
\def\rcs@gdefverb@#1{%
    \endgroup
    \expandafter\gdef\rcs@temp{#1}%
  \endgroup
}
%    \end{macrocode}
% \end{macro}

% \subsection{Keyword Macros}
% \begin{macro}{\rcskwsave}
% Enabled verbatim mode and uses a sub macro to read the arguments afterwards.
%    \begin{macrocode}
\def\rcskwsave{%
  \begingroup
    \rcs@catcodes
    \rcskwsave@readargs
}
%    \end{macrocode}
% \end{macro}

% \begin{macro}{\rcskwsave@readargs}[1]{\$kw: value\$}
% Reads full argument, calls parse submacro and ends catcode changes.
% If \cs{rcskwsave} was called by \cs{rcsid} scans the id keyword by calling the
% scan macro.
%    \begin{macrocode}
\gdef\rcskwsave@readargs#1{%
    \rcskwsave@read#1\relax
  \endgroup
  \if@rcsidsw
    \rcs@ifequal{\rcs@lastkw}{Id}%
      {\ifx\rcskwId\empty\else
        \expandafter
        \rcs@scanId\rcskwId\relax
        \@rcsidswfalse
      \fi}{%
    \rcs@ifequal{\rcs@lastkw}{Header}%
      {\ifx\rcskwHeader\empty\else
        \expandafter
        \rcs@scanId\rcskwHeader\relax
        \@rcsidswfalse
        \fi}{}%
      }%
  \fi
  \ignorespaces
}
%    \end{macrocode}
% \end{macro}

% \begin{macro}{\rcskwsave@read}[1]{keyword line without surrounding \$ \$}
% Reads the full keyword and strips the dollars.
%    \begin{macrocode}
\begingroup
\catcode`\$=12
\gdef\rcskwsave@read $#1$\relax{%
  \rcs@checkcolon#1:\relax
}
\endgroup
%    \end{macrocode}
% \end{macro}

% \begin{macro}{\rcskwsave@parse}[2]{key}{value}
% Parse the keyword and save it away.
%    \begin{macrocode}
\begingroup
\catcode`\$=11
\gdef\rcskwsave@parse$#1:#2${%
  \expandafter\xdef\csname rcskw#1\endcsname{#2}%
}%
\endgroup
%    \end{macrocode}
% \end{macro}

% \begin{macro}{\rcskwdef}[2]{key}{value}
%    \begin{macrocode}
\newcommand{\rcskwdef}[2]{%
  \gdef\rcs@lastkw{#1}%
  \expandafter\xdef\csname rcskw#1\endcsname{#2}%
}
%    \end{macrocode}
% Example: |\rcskwdef{Rev}{2.3}| will define |\rcskwRev| as `|2.3|'.
% \end{macro}

% We define default values for normal keywords. Keyword |Filename| is the name
% given by |Id| and not a real keyword.
%    \begin{macrocode} ^^A TODO: check if needed
\rcskwdef{Rev}{0.0}
\rcskwdef{Date}{}
\rcskwdef{Author}{}
\rcskwdef{Filename}{}
\rcskwdef{HeadURL}{}
%    \end{macrocode}

% \begin{macro}{\rcskw}[1]{keyword name}
% Macro to get keyword value. Just calls \cs{rcskw}\meta{ARGUMENT} where
% the argument interpreted as text. So \eg |\rcskw{Date}| is the same as
% |rcskwDate| but this could be changed later so always use this interface
% to get the keyword values.
%
% \changes{v1.2}{2007/06/22}{Added warning when a wrong, maybe
% misspelled, keyword is given.}
%    \begin{macrocode}
\newcommand{\rcskw}[1]{%
  \@ifundefined{rcskw#1}%
    {\PackageWarning{rcs-multi}{RCS keyword '#1' not defined (typo?)}}%
    {\csname rcskw#1\endcsname}%
}%
%    \end{macrocode}
% \end{macro}
%

% \subsection{Keyword check and strip macros}
% The following macros are used to test whether the given keywords are fully
% expanded or not.
% RCS supports unexpanded keywords as input with or without colon and
% with or without trailing space(s), \ie a:~|$KW$|, b:~|$KW:$| or c:~|$KW: $|.
% To avoid \LaTeX{} syntax errors in this pre-commit state the keyword is
% checked by the following macros. Unexpanded keywords result in an empty value.
% Also leading and trailing spaces are removed.
%
% \begin{macro}{\rcs@checkcolon}[2]{key}{potential value, might be empty}
% Checks if the keyword contains a colon. It is called by \cs{rcskwsave@read}
% with a trailing |:\relax| so that \#2 will be empty if there is no earlier
% colon or will hold the value with this trailing colon otherwise.
% The first case means that the keyword is unexpanded without colon (case a)
% which leads to an empty value. In the second case \cs{rcs@stripcolon} is
% called to strip the colon and surrounding spaces. The final value is
% returned by |\rcs@value|.
%    \begin{macrocode}
\def\rcs@checkcolon#1:#2\relax{%
  \rcs@ifempty{#2}%
    {\rcskwdef{#1}{}}%
    {\rcs@stripcolon#2\relax\rcskwdef{#1}{\rcs@value}}%
}
%    \end{macrocode}
% \end{macro}

% \begin{macro}{\rcs@stripcolon}[1]{potential value}
% Strips the previous added colon (for \cs{rcs@checkcolon}).
% The remaining argument is checked if it's empty (case b) or only a space
% (case c). Otherwise the keyword is expanded and \cs{rcs@stripspace} is
% called to strip the spaces.
%    \begin{macrocode}
\def\rcs@stripcolon#1:\relax{%
  \rcs@ifempty{#1}%
    {\gdef\rcs@value{}}%
    {\rcs@ifequal{#1}{ }%
      {\gdef\rcs@value{}}%
      {\rcs@stripspace#1\relax\relax}%
    }%
}
%    \end{macrocode}
% \end{macro}

% \begin{macro}{\rcs@stripspace}[2]{first character}{rest of string}
% Strips leading space if present and calls \cs{rcs@striptrailingspace} to
% strip the trailing space.
%    \begin{macrocode}
\def\rcs@stripspace#1#2\relax{%
  \rcs@ifequal{#1}{ }%
    {\gdef\rcs@value{#2}}%
    {\rcs@striptrailingspace#1#2\relax}%
}
%    \end{macrocode}
% \end{macro}

% \begin{macro}{\rcs@striptrailingspace}[1]{string}
% Strips trailing space using the macros parameter text. Must be called with
% |\relax| as end marker.
%    \begin{macrocode}
\def\rcs@striptrailingspace#1 \relax{%
  \gdef\rcs@value{#1}%
}
%    \end{macrocode}
% \end{macro}


% \subsection{Date Macros}
% \begin{macro}{\@rcs@scandate}[1]{date}
% Scans data information in Id keyword and saves them in macros.
%    \begin{macrocode}
\def\@rcs@scandate#1{\@rcs@scandate@#1\empty\relax}

\def\@rcs@scandate@#1-#2-#3 #4:#5:#6#7#8\relax{%
  \gdef\rcsfileyear{#1}%
  \gdef\rcsfilemonth{#2}%
  \gdef\rcsfileday{#3}%
  \gdef\rcsfilehour{#4}%
  \gdef\rcsfileminute{#5}%
  \gdef\rcsfilesecond{#6#7}%
}
%    \end{macrocode}
% \end{macro}

% \begin{macro}{\rcspdfdate}
% Returns date in a format needed for |\pdfinfo|.
%    \begin{macrocode}
\def\rcspdfdate{%
  \rcsyear\rcsmonth\rcsday
  \rcshour\rcsminute\rcssecond00'00'%
}
%    \end{macrocode}
% \end{macro}

% \subsection{Mainfile Makros}
% \begin{macro}{\rcssetmainfile}
% Saves the current filename and URL to macros.
% Will be called automatically in the preamble.
% \changes{v1.2}{2007/06/22}{New macro}
%    \begin{macrocode}
\newcommand{\rcssetmainfile}{%
  \xdef\rcsmainfilename{\rcsfilename}%
  \xdef\rcsmainfileurl{\rcsfileurl}%
}
\AtBeginDocument{\rcssetmainfile}
%    \end{macrocode}
% \end{macro}

% \subsection{Register and FullName Macros}
% \begin{macro}{\rcsRegisterAuthor}[2]{author username}{Full Name}
% Saves the author's name by defining |rcs@author@|\meta{username} to it.
%    \begin{macrocode}
\newcommand{\rcsRegisterAuthor}[2]{%
  \expandafter\def\csname rcs@author@#1\endcsname{#2}%
}
%    \end{macrocode}
% \end{macro}

% \begin{macro}{\rcsFullAuthor}
% \begin{macro}{\rcsFullAuthor*}
% We test if the starred or the normal version is used and call the
% appropriate submacro |rcsFullAuthor@star| or |rcsFullAuthor@normal|.
% \changes{v1.2}{2007/06/22}{Macro now returns the username if the full name
% was not registered.}
%    \begin{macrocode}
\newcommand{\rcsFullAuthor}{%
  \@ifnextchar{*}%
    {\rcsFullAuthor@star}%
    {\rcsFullAuthor@normal}%
}%
%    \end{macrocode}
% \end{macro}
% \end{macro}
% \begin{macro}{\rcsFullAuthor@star}[1]{username}
% Both submacros are calling |rcsFullAuthor@| but with different arguments.
% The star macro also removes the star of course.
%    \begin{macrocode}
\def\rcsFullAuthor@star*#1{%
  \edef\rcs@temp{#1}%
  \rcsFullAuthor@{\rcs@temp}{~(\rcs@temp)}%
}%
%    \end{macrocode}
% \end{macro}
% \begin{macro}{\rcsFullAuthor@normal}[1]{username}
%    \begin{macrocode}
\def\rcsFullAuthor@normal#1{%
  \edef\rcs@temp{#1}%
  \rcsFullAuthor@{\rcs@temp}{}%
}%
%    \end{macrocode}
% \end{macro}
% \begin{macro}{\rcsFullAuthor@}[2]{username}{previous defined trailing string}
% |rcsFullAuthor@| now sets the author's full name. Note that |#2| is empty
% when the normal version is called.
%    \begin{macrocode}
\def\rcsFullAuthor@#1#2{%
  \@ifundefined{rcs@author@#1}%
    {#1}%
    {\csname rcs@author@#1\endcsname #2}%
}
%    \end{macrocode}
% \end{macro}

% \begin{macro}{\rcsRegisterRevision}[2]{revision number}{tag name}
% Saves the revision's name or tag by defining
% |rcs@revision@|\meta{revisionnumber} to it.
% \changes{v1.2}{2007/06/22}{New macro}
%    \begin{macrocode}
\newcommand{\rcsRegisterRevision}[2]{%
  \expandafter\def\csname rcs@revision@#1\endcsname{#2}%
}
%    \end{macrocode}
% \end{macro}

% \begin{macro}{\rcsFullRevision}
% \begin{macro}{\rcsFullRevision*}
% We test if the starred or the normal version is used and call the
% appropriate submacro |rcsFullRevision@star| or |rcsFullRevision@normal|.
% \changes{v1.2}{2007/06/22}{New macro}
%    \begin{macrocode}
\newcommand{\rcsFullRevision}{%
  \@ifnextchar{*}%
    {\rcsFullRevision@star}%
    {\rcsFullRevision@normal}%
}
%    \end{macrocode}
% \end{macro}
% \end{macro}
%
% \begin{macro}{\rcsFullRevision@star}[1]{revision number}
% Both submacros are calling |rcsFullRevision@| but with different arguments.
% The star macro also removes the star of course.
%    \begin{macrocode}
\def\rcsFullRevision@star*#1{%
  \edef\rcs@temp{#1}%
  \rcsFullRevision@{\rcs@temp}{~(r\rcs@temp)}%
}
%    \end{macrocode}
% \end{macro}
% \begin{macro}{\rcsFullRevision@normal}[1]{revision number}
%    \begin{macrocode}
\def\rcsFullRevision@normal#1{%
  \edef\rcs@temp{#1}%
  \rcsFullRevision@{\rcs@temp}{}%
}
%    \end{macrocode}
% \end{macro}
% \begin{macro}{\rcsFullRevision@}[2]{revision number}{previous defined trailing
% string}
% |rcsFullRevision@| now sets the revision name. Note that |#2| is empty
% when the normal version is called.
%    \begin{macrocode}
\def\rcsFullRevision@#1#2{%
  \@ifundefined{rcs@revision@#1}%
    {Revision #1}%
    {\csname rcs@revision@#1\endcsname #2}%
}
%    \end{macrocode}
% \end{macro}

% \subsection{Other macros}
% This section contains macros which don't fit in any other section.
%
% \begin{macro}{\rcs}
% Strips the |$ $| around the keyword. A space must be before the final dollar.
% \begin{macrocode}
\providecommand{\rcs}[1]{\@revs#1}
\def\@rcs$#1 ${#1}
%    \end{macrocode}
% \end{macro}

% \begin{macro}{\rcsnolinkurl}[1]{URL}
% This code is taken from the \pkg{hyperref} package and is the definition of
% |\url| just without the part which creates the actual hyperlink. This needs
% of course the \pkg{hyperref} package. A warning is given if it isn't loaded.
% \changes{v1.2}{2007/06/22}{New macro}
%    \begin{macrocode}
%% Adapted from the \url macro of the `hyperref` package.
\DeclareRobustCommand*{\rcsnolinkurl}{%
  \@ifundefined{hyper@normalise}%
    {\PackageWarning{rcs-multi}{Package hyperref is needed for \noexpand
     \rcsnolinkurl.}}%
    {\hyper@normalise\rcsnolinkurl@}%
}%
\def\rcsnolinkurl@#1{\Hurl{#1}}%
%    \end{macrocode}
% \end{macro}

% \begin{macro}{\rcs@getfilename}[1]{URL}
% This macro expands the content using the temporary macro and sets it in front
% of the \csi{rcs@getfilename} sub-macro together with |/{}| to make sure the
% macro does not break at values without directories. A |\relax| is used as
% end marker.
%    \begin{macrocode}
\def\rcs@getfilename#1{%
  \begingroup
    \edef\rcs@temp{#1}%
    \expandafter\@rcs@getfilename\rcs@temp/{}\relax
}%
%    \end{macrocode}
% \end{macro}

% \begin{macro}{\@rcs@getfilename}[2]{URL part before first slash}{URL part after
% first slash}
% Splits the content at the first slash (|/|) and checks if the remainder is
% empty. If so the end marker got reached and the part before the slash is the
% filename which is returned. Otherwise the macro recursively calls itself to
% split the remainder.
%    \begin{macrocode}
\def\@rcs@getfilename#1/#2\relax{%
    \rcs@ifempty{#2}%
      {\endgroup\gdef\rcsfilename{#1}}%
      {\@rcs@getfilename#2\relax}%
}%
%    \end{macrocode}
% \end{macro}

% \subsection{Write to Auxiliary file}
%

% \begin{macro}{\rcs@writeaux}
% This macro writes the |.aux| auxiliary file and is called from a
% |\AtEndDocument| macro later on.
%    \begin{macrocode}
\def\rcs@writeaux{%
    \immediate\write\@mainaux{^^J%
      \noexpand\gdef\noexpand\rcsrev{\@rcs@rev}^^J%
      \noexpand\gdef\noexpand\rcsdate{\@rcs@date}^^J%
      \noexpand\gdef\noexpand\rcsauthor{\@rcs@author}^^J%
      \noexpand\gdef\noexpand\rcsyear{\@rcs@year}^^J%
      \noexpand\gdef\noexpand\rcsmonth{\@rcs@month}^^J%
      \noexpand\gdef\noexpand\rcsday{\@rcs@day}^^J%
      \noexpand\gdef\noexpand\rcshour{\@rcs@hour}^^J%
      \noexpand\gdef\noexpand\rcsminute{\@rcs@minute}^^J%
      \noexpand\gdef\noexpand\rcssecond{\@rcs@second}^^J%
      \noexpand\rcs@gdefverb\noexpand\rcsname{\@rcs@name}^^J%
      \noexpand\rcs@gdefverb\noexpand\rcsurl{\@rcs@url}^^J%
    }%
}
%    \end{macrocode}
% \end{macro}



% At the end of document the values are written to the auxiliary file.
%    \begin{macrocode}
\AtEndDocument{%
  \if@filesw
    \ifx\@rcs@date\empty\else
      \rcs@writeaux
    \fi
  \fi
}
%    \end{macrocode}
%
% \Finale
\endinput
